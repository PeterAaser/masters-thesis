\chapter{Introduction}
\epigraph{They're trying to understand what space is. That's tough for them.
They break distances down into concentrations of chemicals. For them, space is a
range of taste intensities.}{Greg Bear, Blood Music}
%
Countless manhours have been spent improving the design and manufacturing
process of the digital computer, creating more and more complex architectures
capable of operating at ever greater speeds.
%
Neurons are not subject to this top down design philosophy, yet through a
process of self organization they are capable of forming networks capable of
solving complex tasks, with far greater energy efficiency, robustness and
parallellism than any designed processor.
%
Inspired by work done in the field of material computing systems such as the
mecobo platform of nascence [citation needed], living neural networks grown from
human stem cells \emph{in vitro} on \emph{Micro Electrode Arrays} (MEA) are interfaced
with a digital computer, forming a hybrid neuro-digital system.
%
This system utilizes the theoretical framework of \emph{Reservoir Computing} to
help translate between the digital and biological parts of the system, allowing
it to solve simple tasks as a proof of concept.\\ \\
In the 50's and 60's there was much optimism in the burgeoning field of
artificial intelligence. In 1965 H. A. Simon claimed ``machines will be capable,
within twenty years, of doing any work a man can
do.''\cite{vardi_artificial_nodate} , while Marvin Minsky boldly claimed in 1967
that ``Within a generation ... the problem of creating 'artificial intelligence'
will substantially be solved.'' \cite{noauthor_marvin_nodate}.
These claims seem ridiculous now, but at the time the optimism made sense. After
all great strides was made in symbolic programming, computers had finally become
powerful enough to efficiently manipulate logic symbols such as predicates,
negations and implications. With these programs the machine could inferr
conclusions based on previously known information using symbolic logic,
mimmicking the high level reasoning that humans are capable of. While the
computers at the time were only capable of very simple logic inference, such as
concluding the grass was wet based on prior knowledge it had recently rained, it
was believed that the machines ability to infer would grow at the same pace as
processor speeds.
The driving force behind the great leaps made in symbolic computation was the
exponential growth in transistors per area, dubbed moore's law.
This growth allowed processors to maintain exponential performance increases
without straying from the fundamental Von Neumann architecture, sequentially
performing instructions to manipulate memory.
With these performance gains yesteryears intractable problems were turned into
routine operations, and AI it was thought, was no different. Why then did Minsky's
predictions fail so badly?
In short, processors are \emph{designed} in a top down fashion to perform
an instruction set of logical operations, which in turn allows us to design and run programs on them,
for instance programs that operate on logical propositions superficially
similarly to human reasoning.
As complex as they are, each component in a processor has an obvious purpose
(such as adding numbers) and like clockwork these parts work together to execute
the program.
The fundamental difference between the processor and the brain then is that the
as complex as the processor is, its complexity is a linear function of its
components. To understand how a floating point multiplier works it is not
necessary to understand the rest of the processor. This principle of
superposition can not be applied to the brain, attempting to understand how the
brain works by studying a single neuron or cluster neglects the global
interaction, nothing in the brain makes sense in isolation!
\\ \\
In spite of these weaknesses the Von Neumann architecture has been so ubiquitous
that other approaches have been dubbed \cite{Unconventional Computing}.
Unconvential computing, as implied by the name, comes in many forms such as
buckets of water \cite{fernando_pattern_2003}, or blobs of carbon nanotubes
[cite nascence]. 
These approaches seek to utilize the self organizing properties of naturally
occuring processes in physical systems.
% Introduces how computation occurs "naturally"
As an example, consider the effort spent modelling and simulating snow [Cite SIGGRAPH frozen
paper] used in motion pictures such as Disney's frozen. Dozens of machines in
large rendering farms spend weeks rendering the snow in final movie, however if
you bring some dynamite and a helicopter nature will gladly provide you with an
avalanche ``for free''.\\
% Discuss the difficulties of this computation
% Problem: I want to introduce reservoir computing here too, and it's just not
% meshing well with the narrative of reservoir computing. Also the whole
% avalanche thing feels too lengthy as it is now
It's hard to argue that avalanches are computational processes, but they are
vastly parallel and self organized in contrast to the simulated avalanche in
which each particle is meticulously rendered sequentially for each frame.
Avalanches are unlikely to dethrone the Von Neumann approach, just because they
are expensive to calculate does not mean the process can easily be reversed.
%
Still, the interesting properties of avalanches from a computational standpoint
are not different from properties of the processes happening in the primordial
soup, save for one key difference. in the primordial soup life formed, and with
life came evolution.
%
What is the difference between an avalanche and slime mould? While it's
certainly relevant to mention that an avalanche might crush you, from a
computational standpoint the difference is that the process that unfolds in
slime mould serves a \emph{purpose}.
The slime mould seeks food which allows it to
spread and propagate its genes. In short, it responds to its environment,
reacting to sources of food and avoiding danger.
Toffoli [Cite Toffoli] argues that ``Nothing makes
sense in computing except in the light of evolution'', in his eponymous paper,
and he would likely conclude that an avalanche is not a computational process,
but the slime mould is!
%
Indeed, the slime mould can be directly applied to solve computational problems,
such as Adamatzky's application of mould for road planning [Cite Adamatzky Mould
Paper].
%
Whether or not Toffoli is right or not, it's revealing that the cornerstone of
biological evolution, DNA, in itself has impressive computational capabilities,
and as Adleman [Cite DNA comp paper] showed, easily lends itself to solving hard
problems.
%
It is this process that over billions of years has produced the human brain,
capable of extraordinary feats and an excellent s                     arstki
\\ \\
The vast complexity of the human brain has made it a very difficult subject to
study and copy. Rather than understanding the human brain as a whole a more
feasible approach is to understand the underlying processes that allow neural
networks to self-organize into computationally capable networks.
In [Cite DeMarse flight controller] a neural network is grown in an MEA and
interfaced with a flight simulator. [Cite AHDNN] follows a similar approach,
using neurons to control a simple robot. The contribution presented in this
paper builds on this work, but adding RC...

\cleardoublepage

%%% Local Variables:
%%% mode: latex
%%% TeX-master: "../main"
%%% End: