\chapter{Conclusion And Further Work}
\epigraph{The future will be better tomorrow.}
{Vice president Dan Quayle (R)}
The goal of this thesis has been to produce a working proof of concept for
a neuro-digital hybrid.
The system can record, store and broadcast neural recordings which can be
accessed not only by the main experimenting platform, but by anyone with a
working internet connection.
Furthermore, the experimentation platform hosts an online self-reconfigurable
reservoir computer capable of handling reservoirs that are ``moving targets'',
i.e reservoirs whose behavior may change both during and between experiments.
For ease of experimentation a web front-end is available, currently hosted
locally only which provides a convenient and portable frontend for visualizing
and setting up or scheduling experiments.
Lastly, the reservoir computer can request perturbations to be applied to the
reservoir, which is translated into stimuli requests which are broadcasted to
the lab equipment.
Every step in this process is verified to work, save for the last step in which
setting the control registers of the stimulus generatior fails to trigger the
reguested stimuli with the result that the closed loop system is not yet a loop.
In spite of this, the system can be run as intended, verifying that the
filtering of data, rendering of waveforms and simulated robot as well as
perturbation requests are generated as specified.
\section{Further Work}
\subsubsection{Finishing the stated goal of the thesis}
While the lack of results make it hard to draw a scientifically valuable
conclusion it at least makes the further work section easy to write.
The obvious first task is to figure out what is causing stimuli to not work
which is simply a matter of time.
To this end a special DSP debugger chip has been ordered which should make
stimuli working a matter of time.
\subsubsection{Restoring Characteristics of damaged networks}
In addition to running the experiment sketched described in this thesis, the
flexibility and genericity of the pipeline allows other research with little
effort.
This flexibility will be leveraged in a planned experiment where the pipeline
will record certain characteristics of a healthy neural culture and then attempt
to restore characteristics of the network after it has been damaged.
\subsubsection{Exploring capabilities of other reservoirs}
The Socrates project is currently exploring the capabilities of artificial spin
ice (ASI).
Although the project is currently on the simulator stage, the computational
capabilities of ASI have been shown to have many similarities to those of
neurons.
Once the project has produced a physical prototype it will be easy to plug in to
SHODAN, which can provide a head-to-head comparison of neural and ASI
capabilities.
\subsubsection{Modeling the growth rules of neurons}
By learning more about how neurons organize themselves models can be made that
can more closely mimic the growth done by real neural tissue.
This could help neural networks close some of the gap between their current
specific intelligence bringing them at least one step closer to general
intelligence.
%%% Local Variables:
%%% mode: latex
%%% TeX-master: "../main"
%%% End:
