\chapter{Background}
\section{Complex Systems}
% Her følger jeg CS -> RC på samme måte som backgrounden i semproj. Mye mulig at
% det ikke er riktig eller nødvendig.
\emph{
  In this body of work references are made to computations done
  by both artificial and real neurons. To make the distinction
  between these cases clear all computation done by computer
  simulated approximations of neurons will be prefixed as
  artificial.
}\\
Creating a bridge between machine and neural culture is not
useful in itself without a way to interpret the electrical activity measured
from the network, and a way to encode information as electrical pulses to be
transmitted back to the network. To harness the computational power of neural
networks it is necessary to employ a simplified model of neurons capable of
describing how they self-organize into computationally capable networks, but
leaving out the implementation details.
In this endeavor it is very helpful that nature, through the process of
evolution seems predisposed to creating systems that in many ways exhibit
behavior similar to the brain.
Eco-systems, ant-hills and social networks have much in common: They all exhibit
complex non-linear behavior where the global behavior of the system cannot be
traced back to a single part.
Furthermore, their behavior is certainly not designed, but has arisen from
millions of years of evolution.
In these \emph{Complex Systems} positive feedback can loops amplify small
perturbations into cascading effects, changing the entire system, while negative
feedback loops may cause other states to be relatively stable, resulting in
multiple meta-stable states, so called \emph{attractors}, that give the system some
measure of order.\\
The first step in creating a theoretical framework for bridging the gap between
neuron and machine is then to study simpler bio-inspired models exhibiting this
complex behavior, such as \emph{cellular automata}.
A cellular automaton is a model of a single cell that will change its state based only on
its immediate neighbors. They are capable of solving global problems such
as contour-extraction \cite{sipper_emergence_1999}, establishing that local interactions can
produce interesting global behavior.\\
Cellular automata are even sufficiently powerful to express a turing machine,
but as Sipper puts it: ``This is perhaps the quintessential example of a slow
bullet train: embedding a sequential universal Turing machine within the
highly parallel cellular-automaton model.''
Embeddeding turing machines into cellular automatas is of little use, but it's
useful to know that cellular automata are sufficiently powerful if we are to
apply it as a model for the processes governing neural networks.
The real power of cellular automatas as a model for neural networks is how they
model the \emph{phase transitions} in behavior (i.e dynamics).
In Langton's pioneering paper \emph{Computation on the Edge of Chaos} \cite{langton_computation_1990} 
the system dynamics of cellular automata are shown to follow phase transitions
similar to physical matter.
Langton explored the rule space of cellular automata and found that the ratio
between transitions that led to cell death and life had similarities to
temperature in physical systems.
As expected, rules which tended to favor cell death led to static or periodic
systems, while rules favoring life over death led to chaotic systems.
More interstingly is what happened when the rules favored life and death
equally.
In these systems which exists at the border between orderly and chaotic systems
Langton found a \emph{critical} phase where the system was neither chaotic nor
ordered.
It is important to note that Langton did not seek to solve a specific problem
with his automatons, but to explore which automatas capable of supporting
universal computation, hypothesized by Wolfram \cite{wolfram_universality_1984}.
Criticality applies to any dynamic system, not just cellular automata, and the
study of adaptive networks \cite{sayama_modeling_2013} suggests that many
systems exhibit a homeostatic regulation of system dynamics to ensure that it
stays in the critical phase, including neurons
\cite{bornholdt_topological_2000}.
\section{Evolution In Materio}
aaa
\section{Reservoir Computing}
bbb
\section{Neurons As Computers}
% Hva jeg vil få frem:
% En kjapp intro av nevroner
% Hvordan jeg ser nevroner som enkle noder som kommuniserer elektrisk uten å ta
% hensyn til det underliggende elektrokjemiske systemet.
% Hvordan nevroner selv-organiserer seg, og hvordan struktur påvirke adferd, som
% igjen påvirker videre vekst (noe som er veldig veldig interessant!!)
% Beskrive hvordan spiking fører til kjedereaksjoner
% Beskrive hvordan vi kan koble oss på nevronene vha elektriske signaler.
Even a single neuron is an extremely complex system when compared to the
transistor. Together they form vast interconnected networks where information is
exchanged using electrochemical signals. A fundamental difference between
neurons and conventional computers is that these networks have not been
designed, they do not come with a blueprint, and they constantly modify
themselves in response to the environment. Due to the complexities of this
process it is completely necessary to restrict ourselves to a simple model of
the neuron, leaving genetic activations and chemical pathways to the
neurologists and chemists. In order to understand how neurons can assemble into
complex networks capable of thought one must of course understand the underlying
chemical and biological principles, but from a computer scientists perspective
these are mere implementation details necessitated by existing in a physical
universe. Were the physical laws altered the chemical pathways would surely
differ, but the resulting behavior would probably [According to who?] still be
similar to our universe. In this viewpoint, the very essence of neurons is their
ability to self-organize into structures in a bottom-up fashion in a complex
interplay between behavior and form. Following this rationale the neuron will be
seen as a simple node in a network, communicating using electrical pulses,
so-called spikes or action potentials. These spikes are short bursts of
electricity which after firing causes a quick refactory period in which the
firing cell will not respond to stimuli.
When fired these spikes stimulates connected neurons, which may in turn
release their charge causing a signal to cascade in a feedback loop.
These electrical pulses can be seen as the ``language'' of neurons, and by
measuring electrical activity and stimulating the network using electrodes, it
is possible to set up two-way communication between a machine and a cluster of
neurons.

\section{}
\cleardoublepage

%%% Local Variables:
%%% mode: latex
%%% TeX-master: "../main"
%%% End: