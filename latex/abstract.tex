\section*{\Huge Abstract}
\addcontentsline{toc}{chapter}{Abstract}
The human brain is a vastly parallel, energy efficient and robust computing
machine that arises from the self-organizing capabilities of billions of neurons.
In contrast, digital computers are brittle devices whose exponential growth in
processing power is now waning even as moore's law is coming to an end.
The main issues faced by processors is the energy consumption and design
complexity of attempting to parallelize the inherently sequential underlying
von-neumann architecture.\par
This work is part of an interdisciplinary effort to investigate and exploit the
properties of biological substrates, i.e neurons, for computation.
%
A cyborg, short for cybernetic organism, is being created as part of the NTNU
cyborg project, targeting a hybrid biological digital system connected to a
robot, serving as a proof of principle of such a system.
%
The cyborg is a real world robot, controlled by a hybrid neuro-digital system,
consisting on neural tissue cultivated in the lab, an interface between the
analog domain of neurons and digital computer logic through electrical signals,
and a flexible framework for utilizing reservoir computing to provide a bridge
between the digital logic of the robot control system and the dynamics of the
neural tissue.\par
%
The focus of this thesis is the design and implementation of a closed loop
system where a digital computer interfaces with the neural tissue, forming a
two-way bridge allowing the cyborg to learn how to navigate a simulated robot
through a maze with no human intervention.
The theoretical framework of reservoir computing is presented, providing a technique
for communicating with neural cultures which is then used to implement a working
proof of concept system, allowing neural tissue located at the neuroscience lab
at the medical faculty at NTNU to control a small simulated robot remotely over an
internet connection.
All the nessesery software for interfacing with neurons, from the low level
analog-digital level to the high-level interpretation of signals has been
implemented as part of a platform for experimentation which handles networking,
recording and playback of experiments, and the configuration of parameters for
the embedded reservoir computer which is now available as the end product.
A first experimental set-up is included as a working example.
%%% Local Variables:
%%% mode: latex
%%% TeX-master: "../main"
%%% End: