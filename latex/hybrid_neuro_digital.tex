\chapter{Making Of A Cyborg}
This chapter describes the physical components, biological and mechanical, used
to create a cyborg.
\section{Concept}
A biological neural network is grown in a \textit{MEA}, short for \textit{micro electrode
 array} which contains electrodes which interface with neurons using electrical signals.
These MEAs are then \emph{embodied} in a physical robot which is controlled by
the electrical signals as shown in [ref cyborg concept] This system is a
\textit{closed loop}, it does not require any outside intervention, such as a
human controller using a joystick to operate, it's simply driven by the neurons
response to what the sensors of the machine senses.
\section{Platform}
Providing an interface between neurons and a computer allows using neural
network for computation, however it is impractical to move the neural cultures
outside of the laboratory. Rather than moving the cell cultures outside the safe
confines of the incubator, the robot and cell cultures are linked over a network
connection.
Thanks to this decoupling the cell cultures do not have to be embedded in a
physical robot, instead they can be interfaced with any robot connected to the
internet, and even to virtual robots that only exist as a simulation.

Similar network architectures have been implemented, in
[Cite Application] a neuron culture is used to control a simple
wall-avoiding robot as a proof of concept.
In contrast with previous work the robot used in the NTNU cyborg project is a
sophisticated robot which is programmed to move by itself, follow a person, take a selfie
with someone and upload it to facebook, and even perform a secret handshake.
By utilizing an already functioning robot as a host the cyborg project can focus
on extending the capabilities of the robot, making it a true hybrid between
digital and cellular computing.
% \begin{figure}[h!]
%     %\centering
%     \includegraphics[width=\linewidth]{images/kybrg.jpg}
%     \caption{The NTNU cyborg}
%     \label{fig:cyborg}
% \end{figure}
\section{Growing Neurons In Vitro}
The neuron cultures used in the cyborg are being grown in MEAs
at the department of neuroscience.
The MEAs are seeded with neural stem cells of either human or rat origin which
then spontaneously form networks.
At seeding there is no network at all, only a ``soup'' of dissociated neurons
which over the course of several weeks start forming networks.
As the networks starts ``maturing'' a common phenomenon is neurons firing
monotonic spikes automatically.
The activity from these so-called pacemaker neurons can be seen in \ref{fig:pacemaker}.
In the figure each cell corresponds to one of the electrodes as seen in
\ref{fig:cellular_networks}, however at this stage the monotonic spiking
activity tends to be transient, starting and stopping randomly.
\section{Neuron Interfacing Hardware}
The hardware used to interface with neuron cultures for the cyborg is an
\textit{MEA2100} system purchased from multichannel systems. 
The MEA2100 system is built to conduct in-vitro experiments electrically active
cell cultures such as neurons.
The principal components of the MEA2100 systems are:
\section{Micro Electrode Array}
Introduced in the previous chapter, the \textit{MEA} is equipped with an array
of microscopic electrodes capable of sensing and delivering voltages to and from
nearby neurons.
\ref{fig:generic_MEA} shows an empty MEA,
\ref{fig:st_olav_MEA} shows an MEA used by the department of neuroscience with a live neuron culture.
\section{Headstage}
The electrodes of the MEAs are measured and stimulated by the headstage which
contains the necessary high precision electronics needed for microvolt range readings.
\ref{fig:headstage} shows the same type of headstage used in this paper along
with an MEA.
\section{Interface board}
The interface board connects to up to two head-stages and is responsible for interfacing
with the data acquisition computer, as well as auxiliary equipment such as temperature
controls.
The interface board has two modes of operation.
In the first mode the interface board processes and filters data from up to two
headstages as shown in \ref{fig:IFB_regular} which can then be acquired on a normal
computer connected via USB.
In the second mode of operation a Texas instruments TMS320C6454 digital signal
processor is activated which can then be interfaced with using the secondary USB
port as shown in \ref{fig:IFB_DSP}
%%% Local Variables:
%%% mode: latex
%%% TeX-master: "../main"
%%% End: