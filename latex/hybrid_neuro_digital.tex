\chapter{Making Of A Cyborg}
\epigraph{What is real? How do you define 'real'? If you're talking about what
you can feel, what you can smell, what you can taste and see, then 'real' is
simply electrical signals interpreted by your brain.}
{Morpheus to Neo - The Matrix}
The word cyborg is portmanteau of cybernetic organism, a hybrid between the
biological and mechanical.
TODO: Write good text here.
In [Cite AHDNN] researchers created 
\section{Concept}
Shown in figure [rm 3] a conceptual cyborg is shown.
This conceptual cyborg is comprised of three main components: An \emph{MEA}, short
for \emph{Micro Electrode Array} in which a biological neural network is grown.
A \emph{neural interface}, allowing two-way communication between the neural network and
the outside world.
A robotic body, responding to movement commands and equipped with a sensor
allowing it to perceive its environment.
In this concept neural readouts are transformed into a Left and Right signal by
a feed forward neural network, controlling the direction of the robot.
Simultaneously data retrieved from the sensors of the robot are processed in a
feedback processor and fed back to the neural network.
The conceptual cyborg is a closed loop system: The only input to the system is
what the sensors perceive which the cyborg must act upon.\\
The major challenges involved in realizing the cyborg is divided into two major
areas.
Firstly, the infrastructure, comprising of everything from moving data between
computers, to the neurons themselves must be constructed and implemented.
This challenge can in turn be divided into software and hardware, the latter
being the focus of this chapter.
The second challenge is actually configuring the artificial feed forward neural
network such that the cyborg does something that we deem useful.
This task is done in software, and will be covered together with the software
part of the first challenge in the next chapter.
\section{Platform Architecture}
The centerpiece of the cyborg is the neural culture, but for all their
robustness they cannot survive long outside the laboratory, and should therefore
not be physically connected to the robot.
While this decoupling the neural cultures from the physical robot is a practical
necessity, it also has far reaching consequences on the design space, and even
philosophical ramifications.
The decoupling between culture and machine means that the primary focus is to
expose the neural neural interface to the rest of the world, which is done over
network in this project.
With this architecture the robot itself becomes less interesting, it does not
even have to be a physical robot at all, in fact all work presented in this
thesis has been done with a fully virtual robot.
% TODO: mer applications
% Similar network architectures have been implemented, in [Cite Application] a
% neuron culture is used to control a simple wall-avoiding robot as a proof of
% concept.
\section{Wetware}
As opposed to hardware and software, the term wetware describes system
components of biological origin, i.e ``wet'' components.
The wetware of the cyborg is thus the neural networks which are being grown in
MEAs at the department of neuroscience.
The MEAs are seeded with neural stem cells of either human or rat origin which
then spontaneously form networks.
At seeding there is no network at all, only a ``soup'' of dissociated
neurons which over the course of several weeks start forming networks.
As the networks starts ``maturing'' a common phenomenon is neurons firing
monotonic spikes automatically.
The activity from these so-called pacemaker neurons can be seen in
\ref{fig:pacemaker}.
In the figure each cell in the grid corresponds to one of the
electrodes as seen in \ref{fig:cellular_networks}, however at this stage the
monotonic spiking activity tends to be transient, starting and stopping
randomly.

% TODO: Legg til info om complexity, graf fra johannes(?)
\section{Neural Interface}
An \textit{MEA2100} system has been purchased from the lab equipment vendor
multichannel systems GmbH.
This system system is built to conduct in-vitro experiments electrically active
cell cultures such as neurons contained in micro electrode arrays.
Essentially, the MEA2100 provides very precise electrodes for inducing voltage
and measuring.
In addition to voltage and current the equipment comes with temperature controls
allowing neural cultures to survive for prolonged experiments, allowing
experiments on live neurons to last up to 30 minutes.
The essential components of the MEA2100 neural interface as shown in fig ??? are
as follows:

\subsection{Micro Electrode Array}
Shown in fig [Stock photo of MEA] the \textit{MEA}s used in the cyborg project has been procured
from multichannel systems.
These MEAs feature 60 electrodes (with one of these serving as reference
voltage).
Each MEA is seeded with a neural culture, meaning that once seeded an MEA will
be the host of a single culture, each capable of living for over a year.
% TODO: Upon adding figure, move the kool story about frank the tank to the
% subtext
Fig [Microscope pic of Frank] shows a seeded MEA, nicknamed ``Frank'' as a play on frankensteins
monster due to it's tendency to develop \emph{Organoids}, small proto-organs
with structures similar to eyes and other sensor organs.
\subsection{Headstage}
The headstage is responsible for performing the actual readings and stimuli on
the MEA.
The electrodes of the MEAs are measured and stimulated by
the headstage which contains the necessary high precision electronics needed for
microvolt range readings.
Fig [Some hardware pics from the lab] shows the headstage used by the cyborg project.
\subsection{Interface board}
The interface board connects to up to two head-stages
and is responsible for interfacing with the data acquisition computer, as well
as auxiliary equipment such as temperature controls. The interface board has two
modes of operation. In the first mode the interface board processes and filters
data from up to two headstages as shown in \ref{fig:IFB_regular} which can then
be acquired on a normal computer connected via USB. In the second mode of
operation a Texas instruments TMS320C6454 digital signal processor is activated
which can then be interfaced with using the secondary USB port as shown in
\ref{fig:IFB_DSP}
%%% Local Variables:
%%% mode: latex
%%% TeX-master: "../main"
%%% End: