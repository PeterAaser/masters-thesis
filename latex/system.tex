\chapter{An RC Cyborg Platform}
This chapter provides an overview of the components comprising the final system.
fig??? shows an early idealized version of the cyborg which remains the core
focus of the system. In this ideal cyborg an interface connects the biological
nerual network to an ANN readout layer, which in turn controls a robot whose
input is processed and relayed back to the neural network. With this guiding
principle the final system has been designed with the following components.
\begin{description}
\item[Core Reservoir Interface]\mbox{}\\
  Responsible for providing a bdirectional bridge between the reservoir and the
  outside world.
\item[Data Processing]\mbox{}\\
  Processes the signals from the biological neural network, as well as the sensory
  input from the robot to a neuro-compatible format.
\item[Agent Control]\mbox{}\\
  Embodies the biological neural network in a robot, either artificial or real.
\end{description}
These components are enough to satisfy the ideal version of the cyborg, but for
a cyborg to function in a practical setting additional components are necessary: 
\begin{description}
\item[Communication]\mbox{}\\
  While the Core Reservoir Interface provides a bidirectional bridge, it is the
  Communications module that extends that bridge to any machine connected to a
  network.
\item[Recording]\mbox{}\\
  Data from reservoirs is stored in a database, making experiment data
  accessible to any computer connected to the network.
\item[Online Reconfiguration]\mbox{}\\
  The Online Reconfiguration module is responsible for providing the cyborg with
  a filter that interprets the data from the biological neural network which is
  achieved by reconfiguring the filter when the system is running to adapt to
  the current reservoir.
\end{description}
Together these components form a system, whose architecture shown in fig. ???
looks quite different from the ideal cyborg.

% MEAME and SHODAN should be introduced here
% This needs to be backed up with a much better figure
% Consider the below a description of how to make the diagrams etc
\section{A Closed Loop Example}
The final architecture is quite extensive compared to the ideal version, but it
still exists to provide a closed loop embodiment of the neural network. To give
an idea of how the system is used it is necessary with an example run.
\paragraph{Setup}
The user accesses a web interface provided by the main server (SHODAN) and
configures the experiment. This entails setting up a database recording,
selecting filter and parameters and selecting what sort of agent should be running.
\paragraph{Launch}
Once the experiment is configured the main server contacts the lab computer (MEAME)
and requests data acquisition to start. MEAME creates a TCP socket which holds
the data read from the reservoir.
\paragraph{Execution}
After setup is done the 
\section{SHODAN}

\cleardoublepage

%%% Local Variables:
%%% mode: latex
%%% TeX-master: "../main"
%%% End: